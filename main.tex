\documentclass[border = 10pt]{standalone}
\usepackage{tikz}

% Define a variable for the default length
\newcommand{\len}{4}

\begin{document}
\begin{tikzpicture}
  
  % Draw nodes
  % ++(angle;length) controls where the next node is located
  \node (A) {Storm events};
  \path (A) ++(-30:\len) node (B) {Reefs};
  \path (B) ++(30:\len) node (C) {Population of starfish};
  \path (C) ++(45:\len) node (D) {Population of phytoplankton};
  \path (D) ++(45:\len) node (E) {Fertilizer runoff};
  \path (C) ++(-45:\len) node (F) {Population of giant triton snail};
  \path (F) ++(45:\len) node (G) {Fishing};
  
  % Draw arrows. Remove [-latex] to draw line instead of arrow
  \draw[-latex] (A) -- (B);
  \draw[-latex] (B) -- (C);
  \draw[-latex] (D) -- (C);
  \draw[-latex] (E) -- (D);
  \draw[-latex] (F) -- (C);
  \draw[-latex] (G) -- (F);
  
\end{tikzpicture}
\end{document}


% Some references:
% - https://tex.stackexchange.com/questions/27796/position-a-node-a-certain-distance-away-from-another-node-at-a-certain-angle-in

